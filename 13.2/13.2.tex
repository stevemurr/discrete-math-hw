\documentclass{article}
\usepackage[margin=1in]{geometry}
\usepackage{graphicx}
\usepackage[fleqn]{amsmath}
\usepackage{color}
\usepackage{lipsum}
\begin{document}
\setcounter{totalnumber}{5}
   \begin{flushright}
      \Large\textbf{Steven Murr}\\
      \large\textit{HW 13.2} \\
      \large\textit{Problems = \{ 1-8 all\}}
   \end{flushright}
\begin{flushleft}
\makeatletter% Set distance from top of page to first float
\setlength{\@fptop}{5pt}
\makeatother

\setlength\parindent{0pt}1) Draw these state diagrams for the finite-state machinees with these state tables. \\
**See attached paper. \\
~\\
\setlength\parindent{0pt}2) Give the state tables for the finite-state machines with these state diagrams. \\
**See attached paper. \\
~\\\setlength\parindent{0pt}3) Find the output generated from the input string 01110 for the finite-state machine with the state table in: \\
\setlength\parindent{24pt}a) Exercise 1(a). \\
\setlength\parindent{24pt}a) Exercise 1(b). \\
\setlength\parindent{24pt}a) Exercise 1(c). \\
**See attached paper. \\
~\\
\setlength\parindent{0pt}4) Find the output generated from the input string 10001 for the finite-state machine with the state diagram in: \\
~\\
\setlength\parindent{0pt}5) Find the output for each of these input strings when given as input to the finite-state machine in Example 2. \\
\setlength\parindent{24pt}a) 0111 \\
\setlength\parindent{24pt}b) 11011011 \\
\setlength\parindent{24pt}c) 01010101010 \\
**See attached paper. \\
~\\
\setlength\parindent{0pt}6) Find the output for each of these input strings when given as input to the finite-state machine in Example 3. \\
\setlength\parindent{24pt}a) 0000 \\
\setlength\parindent{24pt}b) 101010 \\
\setlength\parindent{24pt}c) 11011100010 \\
**See attached paper. \\
~\\
\setlength\parindent{0pt}7) Construct a finite-state machine that models an old-fashioned soda machine that accepts nickels, dimes, and quarters.  The soda machine accepts change until 35 cents has been put in.  It gives change back for any amount greater than 35 cents.  Then the customer can push buttons to receive either a cola, a root beer, or a ginger ale. \\
**See attached paper. \\
~\\
\setlength\parindent{0pt}8) Construct a finite-state machine that models a newspaper vending machine that has a door that can be opened only after either three dimes (and any number of other coins) or a quarter and a nickel(and any number of other coins) have been inserted.  Once the door can be opneed, the customer opens it and takes a paper, closing the door.  No change is ever returned no matter how much extra money has been inserted.  The next customer starts  with no credit. \\
**See attached paper. \\
\end{flushleft}
\end{document}