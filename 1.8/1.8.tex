\documentclass{article}
\usepackage[margin=1in]{geometry}
\usepackage{graphicx}
\usepackage[fleqn]{amsmath}
\usepackage{color}
\usepackage{lipsum}
\begin{document}
\setcounter{totalnumber}{5}
   \begin{flushright}
      \Large\textbf{Steven Murr}\\
      \large\textit{HW 1.8} \\
      \large\textit{Problems = \{ 1, 6 (do proof by cases), 9, 11, 14, 29, 31, 40 \}}
   \end{flushright}
\begin{flushleft}
\makeatletter% Set distance from top of page to first float
\setlength{\@fptop}{5pt}
\makeatother
\setlength\parindent{0pt}1) Prove that $n^2 + 1 \geq 2^n $ when $n$ is a positive integer with $1 \leq n \leq 4$
\setlength\parindent{24pt} We will do a proof by exhaustion: \\
\setlength\parindent{48pt} a) $(1^2 + 1) = 3 \geq 2^1 = 2 $.  Since 1 squared plus 1 is three and 2 to the first power is 2 and 3 is \\
greater than or equal to 2, the equation has been proven for this case. \\
\setlength\parindent{48pt} b) $(2^2 + 1) = 5 \geq 2^2 = 4 $.  Since 2 squared plus 1 equals 5 and 2 squared is 4 and 5 is greater \\than or equal than 4, the equation has been proven for this case. \\
\setlength\parindent{48pt} c) $(3^2 + 1) = 10 \geq 2^3 = 8 $.  Since 3 to the 2nd power + 1 is 10 and 2 cubed is 8 and 10 is \\greater than or equal to 8, the equation has been proven for this case.\\
\setlength\parindent{48pt} d) $(4^2 + 1) = 3 \geq 2^4 = 16 $.  Since 4 squared + 1 equals 17 and 2 to the 4th power is 16 and 17 is \\greater than or equal to 16, the equation has been proven for this case.\\
~\\
\setlength\parindent{0pt}6) Prove using the notion of without loss generality that 5x + 5y is an odd integer when x and y are integers of opposite parity.\\
\setlength\parindent{24pt} We know that for any integer k, 2k+1 yields an odd integer and 2k yields an even integer.  \\
\setlength\parindent{24pt} If we substitute x and y for 2k+1 and 2j and multiply both by 5 we get: $10k + 10j + 5$\\
\setlength\parindent{24pt} Since the integers x and y, or in the case of the substituted equation, k and j, are of \\opposite parity (meaning one integer is even and the other is odd) we have one case to test.  \\
~\\\setlength\parindent{24pt}\texttt{Case i}: if k is 0 (even) and j is odd (1) then $10(0) + 10(1) + 5 = 15$.  15 is \\an odd integer, thus proving the sum of 5x + 5y is odd given two integers of opposite parity.

~\\Since the two integers must be of opposite parity and we are only working with a domain of integers, \\we have also proven it without loss of generality.  If we created a case involving a negative symbol, the \\output would still always remain odd even if the output may potentially be negative.\\

~\\
\setlength\parindent{0pt}9) Prove that there are 100 consecutive integers that are not perfect squares.  Is your proof constructive or non-constructive? \\
\setlength\parindent{24pt} Let k be an integer.  This represents two numbers that are perfect squares $k^2 and (k+1)^2$.\\
\setlength\parindent{24pt} Since we are searching for 100 consecutive integers between two perfect squares we can use the above\\ to create: $(k+1)^2 - k^2 -1$.  We subtract one because subtracting k squared from k squared + 1 \\would yield 1 which is a perfect square. \\
\setlength\parindent{24pt} Then $(k+1)^2 - k^2 -1 > 100$.  We then square and solve. $k^2 + 2k + 1 - k^2 - 1$.\\
\setlength\parindent{24pt} Everything cancels except for $2k > 100$.  We then divide by 2 finding $k = 50$.\\
\setlength\parindent{24pt} Thus the integers we are looking for must be between $50^2$ and $51^2$.  \\
\setlength\parindent{24pt} $50^2 = 2500$ and $51^2 = 2601$.  We have proven by construction (by showing 100 consecutive integers) \\that there are 100 consecutive integers that are not perfect squares.\\
~\\
\setlength\parindent{0pt}11) Prove that there exists a pair of consecutive integers such that one is a perfect square and one is a perfect cube.\\
\setlength\parindent{24pt} We will use a constructive proof by showing an example of two consecutive integers where one is a\\ perfect square and one is a perfect cube.\\
\setlength\parindent{24pt} $2^3 = 8$ and $3^2 = 9$.  8 and 9 are our consecutive integers.\\
~\\
\setlength\parindent{0pt}14) Prove that if $a$ and $b$ are rational numbers then $a^b$ is also rational.\\
\setlength\parindent{24pt} We will use a constructive proof by showing a specific example where $a^b$ yields an irrational number.\\

\setlength\parindent{24pt} If $a = 2$, which is rational and $b = \frac{1}{2} $ which is rational, then raise a to the bth power, we are left with\\ an irrational number which is equal to $\sqrt{2}$.\\
~\\
\setlength\parindent{0pt}29) Prove that there is no positive integer $n$ such that $n^2 + n^3 = 100$. \\
\setlength\parindent{24pt} We will use a constructive proof by showing specific examples.\\
\setlength\parindent{24pt} We will test values between 1 and 4 since $5^3 = 125$.\\
\setlength\parindent{24pt} $1^2 + 1^3 = 5$\\
\setlength\parindent{24pt} $2^2 + 2^3 = 12$ \\
\setlength\parindent{24pt} $3^2 + 3^3 = 36$\\
\setlength\parindent{24pt} $4^2 + 4^3 = 80$ \\
\setlength\parindent{24pt} Since all integers above 4 will exceed 100, we have shown that there are no positive integers such \\that $n^2 + n^3 $ will equal 100.\\
~\\
\setlength\parindent{0pt}31) Prove that there are no positive integers x and y to the equation $x^4 + y^4 = 625$. \\

\setlength\parindent{24pt} Since $5^4 = 625$ we know that $x$ and $y$ must be less than 5.\\
\setlength\parindent{24pt} If both x and y were 4 then $4^4 + 4^4 = 512$ which is less than 625.\\
\setlength\parindent{24pt} Thus the sum of the integers x and y raised to the 4th power don't equal 625.\\

~\\
\setlength\parindent{0pt}40) Verify the $3x+1$ conjecture for the following integers.\\
\setlength\parindent{24pt} The conjecture states that while the starting number isn't 1, if it's even then divide it by 2 and if \\the number is odd multiply it by 3 and add 1 to it.\\
\setlength\parindent{24pt}a) 16\\
\setlength\parindent{48pt} 16 / 2 = 8 / 2 = 4 / 2 = 2 / 2 = 1 \\
\setlength\parindent{24pt}b) 11 \\
\setlength\parindent{48pt} 3(11)+1 = 34 / 2 = 17 = 3(17)+1 = 52 / 2 = 26 / 2 = 13 = 3(13) + 1 = 40 / 2 \\= 20 / 2 = 10 / 2 = 5 = 5(3)+1 = 16 /2 = 8 / 2 = 4 / 2 = 2 / 2 = 1.\\
\setlength\parindent{24pt}b) 35 \\
\setlength\parindent{48pt} 35 = 3(35) + 1 = 106 / 2 = 53 = 3(53)+1 = 160 / 2 = 80 / 2 = 40 / 2 = 20 / 2 = \\10 / 2 = 5 = 3(5) + 1 = 16 / 2  = 8 / 2 = 4 / 2 = 2 / 2 = 1.\\
\setlength\parindent{24pt}d) 113 \\
\setlength\parindent{48pt} 113 = 3(113)+1 = 340 / 2 = 170 / 2 = 85 = 3(85) + 1 = 256 / 2 = 128 / 2 = 64 / 2 = \\32 / 2 = 16 / 2 = 8 / 2  = 4 / 2 = 2 / 2 = 1.





\end{flushleft}
\end{document}