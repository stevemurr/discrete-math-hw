\documentclass{article}
\usepackage[margin=1in]{geometry}
\usepackage{graphicx}
\usepackage[fleqn]{amsmath}
\usepackage{color}
\usepackage{lipsum}
\begin{document}
\setcounter{totalnumber}{5}
   \begin{flushright}
      \Large\textbf{Steven Murr}\\
      \large\textit{HW 1.8 REDO}
   \end{flushright}
\begin{flushleft}
\makeatletter% Set distance from top of page to first float
\setlength{\@fptop}{5pt}
\makeatother

\setlength\parindent{0pt}6) Prove using proof by cases that $5x + 5y$ is an odd integer when x and y are integers of opposite parity. \\
\setlength\parindent{24pt}We know that for any integer k and j, $2k + 1$ will yield an odd integer and $2j$ will yield an even integer.  We have two cases to prove.  One in which x is odd and y is even and one in which x is even and y is odd.  \\
~\\
\setlength\parindent{48pt} case i: If x is odd and y is even:\\
~\\
\setlength\parindent{48pt} If we substitute x and y for $2k + 1$ and $2j$ and multiply both out we get:  $10k + 10j + 5$.  \\We are looking to put the equation into the form $2k + 1$ to yield an odd integer.  \\If we make the equation $10k + 10j + 5$ look like $10k + 10j + 4 + 1$ we are now able\\ to factor out a 2 and leave a one outside the parenthesis.  The equation then becomes:  \\
\setlength\parindent{48pt} $2(5k + 5j + 2)+ 1$ \\
\setlength\parindent{48pt} Using the products of integers and sum of integers rule we know that $5k + 5j + 2$ will yield an \\integer.  We are now able to rewrite the equation in the form. \\
\setlength\parindent{48pt} $2(integer) + 1$ which satisfies $5x + 5y = $ an odd integer. \\
~\\
\setlength\parindent{48pt} case ii: If x is even and y is odd: \\
~\\
\setlength\parindent{48pt} $5(2j) + 2(2k+1) = 10j + 10 k + 5$ \\
\setlength\parindent{48pt} Similarly we want the equation to be of the form $2k+1$ \\
\setlength\parindent{48pt} If we rewrite the equation to look like $10j + 10k + 4 + 1$ we are now able to factor out a two \\while still leaving a one on the outside. \\
\setlength\parindent{48pt} After factoring it becomes $2(5j + 5y + 2) + 1$.  \\Due to the products of integers being integers and the sum of integers being integers we can \\rewrite the equation of the form:
\setlength\parindent{48pt} $2(integer) + 1$\\
~\\
\setlength\parindent{48pt} We have proven our two cases.


\end{flushleft}
\end{document}