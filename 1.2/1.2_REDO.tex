
\documentclass{article}
\usepackage[margin=1in]{geometry}
\usepackage{graphicx}
\usepackage[fleqn]{amsmath}
\begin{document}
   \begin{flushright}
      \Large\textbf{Steven Murr}\\
      \large\textit{HW 1.2 - REDO}
   \end{flushright}
\begin{flushleft}
\setlength\parindent{0pt}20) A says "The two of us are both knights" and B says "A is a knave."
\\
~\\If A is a knight, and the statement "The two of us are both knights" is true, then B would have to be a knight and B's statement "A is a knave" would have to be true.  If B's statement is true, it contradicts A's statement about both of them being Knights.  \\

~\\If A is a Knave, and the statement "The two of us are both knights" is false, and if B is a Knave, then B's statement "A is a knave" would be false meaning A is a Knight, but Knights only tell the truth, leading to a paradox.\\
~\\
If A is a Knave, and the statement "The two of us are both knights" is false, and if B is a Knight, then B's statement "A is a Knave" would be true, satisfying the system of Knight's telling the truth and Knaves lying.
\end{flushleft}
\end{document}