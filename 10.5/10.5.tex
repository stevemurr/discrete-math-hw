\documentclass{article}
\usepackage[margin=1in]{geometry}
\usepackage{graphicx}
\usepackage[fleqn]{amsmath}
\usepackage{color}
\usepackage{lipsum}
\begin{document}
\setcounter{totalnumber}{5}
   \begin{flushright}
      \Large\textbf{Steven Murr}\\
      \large\textit{HW 10.5} \\
      \large\textit{ Problems = \{ 1,2,5,6,10,13,15,26,30,31,39,40\} }
   \end{flushright}
\begin{flushleft}
\makeatletter% Set distance from top of page to first float
\setlength{\@fptop}{5pt}
\makeatother
\setlength\parindent{0pt}1) In exercises 1-8 determine whether the given graph has an Euler circuit.  Construct such a circuit when one exists.  if no Euler circuit exists, determine whether the graph has an Euler path and construct a path if one exists. \\
**See attached sheet.
~\\
\setlength\parindent{0pt}10) Can someone cross all the bridges shown in this map exactly once and return to the starting point? \\ 
**See attached sheet. \\
~\\
\setlength\parindent{0pt}13) In exercises 13-15 determine whether the picture shown can be drawn with a pencil in a continuous motion without lifting the pencil or retracing part of the picture. \\
~\\
\setlength\parindent{0pt}26) For which values of n do these graphs have a Euler circuit? \\
\setlength\parindent{24pt}a) $K_n$ \\4
\setlength\parindent{48pt} When n is $\geq 2$ the graph is a Euler circuit if all vertices have an even degree.  This is confirmed \\by theorem 1.\\ 
\setlength\parindent{24pt}$C_n$ \\
\setlength\parindent{48pt} When n is $\geq 3$ all cycles with be/have an Eulerian Circuit.  Every vertex has a degree of 2 in a \\cycle and therefore all vertices are even.   \\
\setlength\parindent{24pt}$W_n$ \\
\setlength\parindent{48pt} Wheels can't have Eulerian Circuits but they are capable of having Eulerian Path's.  All vertices \\have odd degrees.\\
\setlength\parindent{24pt} $Q_n$ \\
\setlength\parindent{48pt} When n is  2, $Q_n$ is a square and all vertices have degree 2.  In $Q_n$ when n is 3, all vertices have \\degree 3.  It then makes sense that as n increases all vertices have degree n therefore whenever n \\is even there will be a Eulerian Circuit and not when n is odd. \\
~\\
\setlength\parindent{0pt}30) In Exercises 30-36 determine whether the given graph has a Hamilton circuit. \\

\setlength\parindent{0pt}39) Does the graph in Exercise 32 have a Hamilton path?  If so, find such a path.  If it does not, give an argument to show why no such path exists. \\
\setlength\parindent{24pt}It does have a Hamilton path.  See attached paper. \\
~\\
\setlength\parindent{0pt}40) Does the graph in Exercise 33 have a Hamilton path?  If so, find such a path.  If it doesn't not, give an argument to show why no such path exists. \\
\setlength\parindent{24pt}It does not have a Hamilton Path.  See attached paper. \\


\end{flushleft}
\end{document}