\documentclass{article}
\usepackage[margin=1in]{geometry}
\usepackage{graphicx}
\usepackage[fleqn]{amsmath}
\usepackage{color}
\usepackage{lipsum}
\begin{document}
\setcounter{totalnumber}{5}
   \begin{flushright}
      \Large\textbf{Steven Murr}\\
      \large\textit{HW 5.3} \\
      \large\textit{ Problems = \{ 1,4,7,9,10,15,18,20,49,57 \}}
   \end{flushright}
\begin{flushleft}
\makeatletter% Set distance from top of page to first float
\setlength{\@fptop}{5pt}
\makeatother
\setlength\parindent{0pt}1) Find $f(1), f(2), f(3),$ and $f(4)$ if $f(n)$ is defined recursively by $f(0) = 1$ and for $n = 0,1,2,...$ \\
\setlength\parindent{24pt}a) $f(n+1) = f(n) + 2$
\begin{align*}
f(0) &= 1 \\
f(0+1) &= 1 + 2 = 3 \\
f(1 + 1) &= 3 + 2 = 5 \\
f(2 + 1) &= 5 + 2 = 7 \\
f(3 + 1) &= 7 + 2 = 9
\end{align*}
\setlength\parindent{24pt}b)$f(n+1) = 3f(n)$
\begin{align*}
f(0) &= 1 \\
f(0 + 1) &= 3(1) = 3 \\
f(1 + 1) &= 3(3) = 9 \\
f(2 + 1) &= 3(9) = 27 \\
f(3 + 1) &= 3(27) = 81
\end{align*}
\setlength\parindent{24pt}c) $f(n+1) = 2^{f(n)}$
\begin{align*}
f(0) &= 1 \\
f(0 + 1) &= 2^1 = 2 \\
f(1 + 1) &= 2^2 = 4 \\
f(2 + 1) &= 2^4 = 16 \\
f(3 + 1) &= 2^{16} = 65536
\end{align*}
\setlength\parindent{24pt}d) $f(n+1) = f(n)^2 + f(n) + 1$
\begin{align*}
f(0) &= 1 \\
f(0 + 1) &= 1^2 + 1 + 1 = 3 \\
f(1 + 1) &= 3^2 + 3 + 1 = 13 \\
f(2 + 1) &= 13^2 + 13 + 1 = 183 \\
f(3 + 1) &= 183^2 + 183 + 1 = 33673 
\end{align*}

\setlength\parindent{0pt}Find $f(2), f(3), f(4),$ and $f(5)$ if $f$ is defined recursively by $f(0) = f(1) =1 $and for $n = 1,2,...$ \\
\setlength\parindent{24pt}a) $f(n+1) = f(n) - f(n-1)$
\begin{align*}
f(0) &= f(1) = 1 \\
f(0 + 1) &= 1 - 0 = 1 \\
f(1 + 1) &= 1 - 0 = 1 \\
f(2 + 1) &= 1 - 0 = 1 \\
f(3 + 1) &= 1 - 0 = 1 \\
f(4 + 1) &= 1 - 0 = 1
\end{align*}
\setlength\parindent{24pt}b) $f(n+1) = f(n)f(n-1)$ 
\begin{align*}
f(0) &= f(1) = 1 \\
f(0 + 1) &= 1\cdot 0 = 0 \\
f(1 + 1) &= 0\cdot 0 = 0 \\ 
f(2 + 1) &= 0\cdot 0 = 0 \\ 
f(3 + 1) &= 0\cdot 0 = 0 \\
f(4 + 1) &= 0\cdot 0 = 0
\end{align*}
\setlength\parindent{24pt}c) $f(n+1) = f(n)^2 + f(n-1)^3$ 
\begin{align*}
f(0) &= f(1) = 1 \\
f(0 + 1) &= 1^2 + 0 = 1 \\
f(1 + 1) &= 1^2 + 0 = 1 \\
f(2 + 1) &= 1^2 + 0 = 1 \\
f(3 + 1) &= 1^2 + 0 = 1 \\
f(4 + 1) &= 1^2 + 0 = 1
\end{align*}
\setlength\parindent{24pt}d) $f(n+1) = f(n) / f(n-1)$ 
\begin{align*}
f(0) &= f(1) = 1 \\
f(0 +1 ) &= 1 / 0 = undefined \\
f(1 + 1) &= und / und \\
f(2 + 1) &= und / und \\
f(3 + 1) &= und / und \\
f(4 + 1) &= und / und 
\end{align*}
\setlength\parindent{0pt}7) Give a recursive definition of the sequence ${a_n}, n = 1,2,3 $ if: \\
\setlength\parindent{24pt}a) $a_n = 6n$
\begin{align*}
a_n &= 6n \\
a_1 &= 6 \\
a_{n+1} &= 6(n+1) \\
a_{n+1} &= 6n + 6 \\ 
&= a_n + 6
\end{align*}
\setlength\parindent{24pt}b) $a_n = 2n + 1$
\begin{align*}
a_{n} &= 2n + 1 \\
a_1 &= 3  - 2(1) + 1\\
a_{n+1} &= 2(n+1) + 1 = 2n + 2 + 1 \\
&= 2n + 1 + 2 = a_n + 2 
\end{align*}
\setlength\parindent{24pt}c) $a_n = 10^n$
\begin{align*}
a_n &= 10^n \\
a_1 &= 10 \\
a_{n+1} &= 10^{n+1} \\
&= 10 \cdot 10^n \\
&= a_n \cdot 10 
\end{align*}
\setlength\parindent{24pt}d) $a_n = 5$ 
\begin{align*}
a_n &= 5 \\
a_1 &= 5 \\
a_{n+1} &= 5  \\
&= a_n 
\end{align*}
**(Since $a_n$ is referenced as a constant all values will point to 5) \\
~\\
\setlength\parindent{0pt}9) Let F be the function such that $F(n)$ is the sum of the first n positive integers.  Give a recursive definition of $F(n)$
\begin{align*}
a_0 &= 0 \\
f(n) &= n + a_{n-1} : n > 1
\end{align*}
\setlength\parindent{0pt}10) Give a recursive definition of $P_m(n)$, the sum of the integer $m$ an the nonnegative integer $n$. 
\begin{align*}
P_0 &= 0 \\
P_m(n) &= p_m + n
\end{align*}
\setlength\parindent{0pt}15) Show that $f_0f_1 + f_1+f_2+... +f_{2n-1}f{2n} = f\frac{2}{2n}$ \\
Basis Step: $f_0f_1+f_1f_2 = 0 \cdot 1 + 1 \cdot 1 = 1^2 = f\frac{2}{2}$ \\
Inductive Step: $f_0f_1 + f_1f_2 + ... + f_{2k-1}f_{2k} = f\frac{2}{2k}$ Then $f_0f_1 + f_1f_2 + ... + f_{2k-1}f_{2k} + f{2k}f{2k+1} + f{2k+1} + f_{2k+2} = f\frac{2}{2k}+ f_{2k}f{2k+1} + f_{2k+1}f_{2k+2} = f_{2k}(f_{2k} + f_{2k+1} + f_{2k+1}f_{2k+2} = f_{2k}f_{2k+2}+f_{2k+1}f_{2k+2} = (f_{2k} + f_{2k+1})f_{2k+2} = f\frac{2}{2k+2}$ 
~\\
\setlength\parindent{0pt}18) Let \\
\[
\left(
\begin{array}{ccc}
  1&   1&   \\
  1&   0& 
\end{array}
\right)
\]
Show that $A^n = $ \\
\[
\left(
\begin{array}{ccc}
  f_{n+1}&   f_n&   \\
  f_n&   f_{n-1}&   \\
  &   &   
\end{array}
\right)
\]
**I'll be coming to your office hours on thursday for help with this problem.
~\\
\setlength\parindent{0pt}49) Show that $A(m,2) = 4$ whenever $m \geq 1$ \\
\setlength\parindent{24pt}Basis Case: Assume P(1) or m = 1.
\begin{align*}
A(1, 2) &- 4th case \\
A(1-1, A(1, 2-1)) &- 3rd case \\
A(0, 2) &- 1st case, 2(2) \\
&= 4
\end{align*}
Inductive Step:
\begin{align*}
&= A(m, 2) \\
&= A(m+1,2) \\
&= A(m, A(m+1, 1)) \\
&= A(m, 2) = 4
\end{align*}
**Whenever m is greater then or equal to one and n is greater then or equal to two, it will always trigger the fourth step causing m to decrement until m = 0 is reached. \\
~\\
\setlength\parindent{0pt}57) Use strong induction to prove that a function F defined by specifying F(0) and a rule for obtaining F(n+1) from the values F(k) for k = 0,1,2,..., n is well defined. \\
Basis Step: F(0) = 0 is true \\
Inductive Step: F(k) is true since f(0) = 0 then f(0+1) or f(1) = 1 when k < n(any integer).  \\
This is a well defined function because recursively defined functions are well defined.  \\


\end{flushleft}
\end{document}