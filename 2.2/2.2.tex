\documentclass{article}
\usepackage[margin=1in]{geometry}
\usepackage{graphicx}
\usepackage[fleqn]{amsmath}
\usepackage{color}
\usepackage{lipsum}
\begin{document}
\setcounter{totalnumber}{5}
   \begin{flushright}
      \Large\textbf{Steven Murr}\\
      \large\textit{HW 2.2} \\
      \large\textit{Problems = \{ 1, 2, 3, 14, 15, 16, 19, 25, 32, 33, 34, 52, 53, 54 \}}
   \end{flushright}
\begin{flushleft}
\makeatletter% Set distance from top of page to first float
\setlength{\@fptop}{5pt}
\makeatother

\setlength\parindent{0pt}1) Let A be the set of students who live within one mile of school and let B be the set of students who walk to classes.  Describe the students in each of these sets. \\
\setlength\parindent{24pt}a) $A \cap B$ \\
\setlength\parindent{48pt} The intersection of A and B is the set of students who live within one mile of school AND \\students who walk to classes. \\ 
\setlength\parindent{24pt}b) $A \cup B$ \\
\setlength\parindent{48pt} The union of A and B is the set of students who live within one mile of school OR \\students who walk to classes. \\ 
\setlength\parindent{24pt}c) $A - B$ \\
\setlength\parindent{48pt} The difference of A and B is the set of students who live within one mile of school who are NOT \\students who walk to classes.\\
\setlength\parindent{24pt}d) $B - A$ \\
\setlength\parindent{48pt} The difference between B and A is the set of students who walk to classes who are NOT \\students who live within one mile of school. \\

~\\
\setlength\parindent{0pt}2) Suppose that A is the set of sophomores at your school and B is the set of students in discrete mathematics at your school.  Express each of these sets in terms of A and B.\\
\setlength\parindent{24pt}a) The set of sophomores taking discrete mathematics in your school. \\
\setlength\parindent{48pt} $A \cap B$ \\
\setlength\parindent{24pt}b) The set of sophomores at your school who are not taking discrete mathematics. \\
\setlength\parindent{48pt} $A - (A \cap B)$ \\
\setlength\parindent{24pt}c) The set of students at your school who either are sophomores or are taking discrete mathematics. \\
\setlength\parindent{48pt} $A \cup B$\\
\setlength\parindent{24pt}d) The set of students at your school who either are not sophomores or are not taking \\discrete mathematics. \\
\setlength\parindent{48pt} $\overline A \cup \overline B$ \\ 

~\\

\setlength\parindent{0pt}3) Let $A = \{ 1,2,3,4,5 \} $ and $B = \{0,3,6\} $.  Find the following: \\
\setlength\parindent{24pt}a) $A \cup B$ \\
\setlength\parindent{48pt} $A \cup B = \{ 0,1,2,3,4,5,6\}$ \\
\setlength\parindent{24pt}b) $A \cap B$ \\ 
\setlength\parindent{48pt} $A \cap B = \{3\}$ \\
\setlength\parindent{24pt}c) $A - B$ \\
\setlength\parindent{48pt} $A - B = \{ 1,2,4,5 \}$ \\
\setlength\parindent{24pt}d) $B - A$\\
\setlength\parindent{48pt} $B - A = \{ 0, 6\}$ \\ 

~\\
\setlength\parindent{0pt}14) Find the sets A and B if $A - B = \{1,5,7,8\}, B - A = \{2,10\}$ and $A \cap B = \{3,6,9\}$\\
\setlength\parindent{24pt} $A = \{ 1,3,5,6,7,8,9\}$ and $B = \{2,3,6,9,10\}$ \\ 
\setlength\parindent{24pt}Starting with $A \cap B$ we know those values are in both A and B.\\
\setlength\parindent{24pt}We then know that the set A - B will be all the values not in B.  So we find A by: \\
\setlength\parindent{24pt}$(A \cap B) \cup (A - B) = A$ \\
\setlength\parindent{24pt}$(A \cap B) \cup (B - A) = B$ \\
~\\
\setlength\parindent{0pt}15) Prove the second Demorgan Law in Table 1 by showing that if A and B are sets then $\overline{A \cup B} = \overline{A} \cap \overline{B}$ \\
\setlength\parindent{24pt}a) By showing each side is a subset of the other side. \\
\setlength\parindent{48pt} $\overline{A \cup B} = \overline{A} \cap \overline{B}$ \\
\setlength\parindent{48pt} $x \in \overline{A \cup B}$ then $x \notin A \cup B$   \\ 
\setlength\parindent{48pt} Meaning $x \notin A$ AND $x \notin B$ since x is not in the union it will not be in the sets.\\
\setlength\parindent{48pt} If $x \notin A$ or $x \notin B$ it will be in the intersection of $\overline{A} \cap \overline{B}$\\
\setlength\parindent{48pt} Since both sides are equivalent they will be subsets of the other side because all elements x \\are present in both sets.\\
\setlength\parindent{24pt}b) Using a membership table. \\
\setlength\parindent{48pt} See handwritten figure attached to the back of this homework. \\

~\\
\setlength\parindent{0pt}16) Let A and B be sets.  Show that: \\
\setlength\parindent{24pt}a) $(A \cap B) \subseteq A$ \\
\setlength\parindent{48pt} If $x \in A$ and $x \in B$ then the intersection of A and B will be a subset of set A. \\
\setlength\parindent{24pt}b) $A \subseteq (A \cup B)$\\
\setlength\parindent{48pt} A will be a subset of $A \cup B$ because if $x \in A$ then it will be present in the union of A with B.\\
\setlength\parindent{24pt}c) $A - B \subseteq A$ \\
\setlength\parindent{48pt} If $x \notin B$ on the left side, and $x \in A$ then we can simplify by showing $A \subseteq A$.  All \\subsets are subsets of themselves.\\
\setlength\parindent{24pt}d) $A \cap (B - A) = \o$ \\
\setlength\parindent{48pt} If $x \in A$ on the left side and $x \not in B$ on the right due to the subtraction of A from B, then the \\two sets will have no shared values and thus will only have the empty set in the union. \\
\setlength\parindent{24pt}e) $A \cup (B - A) = A \cup B$\\
\setlength\parindent{48pt} If $x \in A$ and $x \notin A$ on the right, then $x \in A$ and $x \in B$ in the union of A and B. \\

~\\
\setlength\parindent{0pt}19) Show that if A and B are sets, then: \\
\setlength\parindent{24pt}a) $A - B = A \cap \overline{B}$ \\
\setlength\parindent{48pt} If $A - B$ then $B \notin A$ \\
\setlength\parindent{48pt} If $B \notin A$ then $\overline{B} \in A$ \\ 
\setlength\parindent{48pt} Therefore, $A \cap \overline{B}$ is equivalent to $A - B$ \\
\setlength\parindent{24pt}b) $(A \cap B) \cup (A \cap \overline{B}) = A$\\
\setlength\parindent{48pt} See handwritten membership table. \\

~\\
\setlength\parindent{0pt}25) Let $A = \{0,2,4,6,8,10\}, B = \{0,1,2,3,4,5,6\}, $ and $ C = \{4,5,6,7,8,9,10\}$ Find: \\
\setlength\parindent{24pt}a) $A \cap B \cap C$ \\
\setlength\parindent{48pt} $\{ 2, 4, 6 \}$ \\
\setlength\parindent{24pt}b) $A \cup B \cup C$ \\
\setlength\parindent{48pt} $\{ 0,1,2,3,4,5,6,7,8,9,10 \}$ \\
\setlength\parindent{24pt}c) $(A \cup B) \cap C$ \\
\setlength\parindent{48pt} $\{ 4,5,6,8,9,10 \}$ \\
\setlength\parindent{24pt}d) $(A \cap B) \cup C$ \\
\setlength\parindent{48pt} $\{ 0,2,4,5,6,7,8,9,10 \}$ \\

~\\
\setlength\parindent{0pt}32) Find the symmetric difference of \{ 1, 3, 5 \} and \{ 1, 2, 3 \}\\
\setlength\parindent{24pt} \{ 2, 5 \} \\
\setlength\parindent{0pt}33) Find the symmetric difference of the set of computer science majors at a school and the set of mathematics majors at this school. \\
\setlength\parindent{24pt} The symmetric difference are the students who are either computer science majors or \\mathematics majors.  \\

~\\
\setlength\parindent{0pt}34) Draw a Venn diagram for the symmetric difference of the sets A and B. \\

~\\
\setlength\parindent{0pt}52) Suppose that the universal set is $U = \{ 1,2,3,4,5,6,7,8,9,10 \}$.  Express each of these sets with bit strings where the with bit in the string is 1 if i is in the set and 0 otherwise.\\

\setlength\parindent{24pt}a) \{ 3,4,5 \} \\
\setlength\parindent{48pt} $ 0 0 1 1 1 0 0 0 0 0$ \\
\setlength\parindent{24pt}b) \{1,3,6,10\} \\
\setlength\parindent{48pt} $ 1 0 1 0 0 1 0 0 0 1$ \\
\setlength\parindent{24pt}c) \{2,3,4,7,8,9\} \\
\setlength\parindent{48pt} $ 0 1 1 1 0 0 1 1 1 0$ \\

~\\
\setlength\parindent{0pt}53) Using the same universal set as the last problem, find the set specified by each of these bit strings. \\
\setlength\parindent{24pt}a) 11 1100 1111 \\
\setlength\parindent{48pt} \{ 1,2,3,4,7,8,9,10 \}\\
\setlength\parindent{24pt}b) 01 0111 1000 \\
\setlength\parindent{48pt} \{ 2,4,5,6,7 \}\\
\setlength\parindent{24pt}c) 10 0000 0001 \\
\setlength\parindent{48pt} \{ 1, 10 \} \\

~\\
\setlength\parindent{0pt}54) What subsets of a finite universal set do these bit strings represent? \\
\setlength\parindent{24pt}a) The string with all zeros\\
\setlength\parindent{48pt} The null set. \\
\setlength\parindent{24pt}b) The string with all ones. \\
\setlength\parindent{48pt} It represents the universal set itself. \\


\end{flushleft}
\end{document}