\documentclass{article}
\usepackage[margin=1in]{geometry}
\usepackage{graphicx}
\usepackage[fleqn]{amsmath}
\usepackage{color}
\usepackage{lipsum}
\begin{document}
\setcounter{totalnumber}{5}
   \begin{flushright}
      \Large\textbf{Steven Murr}\\
      \large\textit{Class Notes 3.12}
   \end{flushright}
\begin{flushleft}
\makeatletter% Set distance from top of page to first float
\setlength{\@fptop}{5pt}
\makeatother
5.3 Cont. \\
Ackermans Function \\
~\\
It is a two variable recursive function.  Variables used are $m $ and $n$.  It is piece wise defined.\\
~\\
$A: N x N \rightarrow N$ \\
Takes a natural number of pairs and produces one natural number as output \\ 
Defined by $A(m,n)$ \\
\begin{align*}
1) 2n, if m &= 0 \\
2) 0, if m \geq 1 and n &= 0 \\
3) 2, if m\geq 1 and n &= 1 \\
4) A(m-1,A(m, n-1), if m \geq 1, and n \geq 2
\end{align*}
Compute ackermans function A(m,n): \\
a) a(0,6) \\
Since $m = 0$, then 2(6) thus 12.\\
b) a(6,0) \\
Since $n = 0$ then it equals 0 \\
c) a(6,1) \\
Since m is greater then one and n is equal to 1 then the answer is 2 \\
d) a(1, 2) \\
Since m is greater then or equal to 1 and n is greater then or equal to 2. \\
A(1-1, A(1, 2,1)) \\
2(1)

Ackermans function is interesting because of how fast it grows.  \\
Claim $A(2,5) = 2^65536$ \\
$A(3,6)$ is even bigger \\
Used to test compilers for recursion handling because its so big. \\

Proof involving Ackermans function and induction. \\
Fact(to be proved in HW 5.3) \\
A(1,n) is always $2^n$.  When n is any natural number. \\
~\\
Prove: $A(2, n) = 2^n$ for any integer n $\geq $ 1.\\
$A(2,5)$ is an example of this. \\
Notation: \\

We are doing repeated exponentiation not repeated multiplication. \\

\begin{align*}

\end{align*}

\end{flushleft}
\end{document}