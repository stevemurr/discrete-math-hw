\documentclass{article}
\usepackage[margin=1in]{geometry}
\usepackage{graphicx}
\usepackage[fleqn]{amsmath}
\usepackage{color}
\usepackage{lipsum}
\begin{document}
\setcounter{totalnumber}{5}
   \begin{flushright}
      \Large\textbf{Steven Murr}\\
      \large\textit{HW 2.1 REDO}
   \end{flushright}
\begin{flushleft}
\makeatletter% Set distance from top of page to first float
\setlength{\@fptop}{5pt}
\makeatother

\setlength\parindent{0pt}6) Suppose that A = \{ 2, 4, 6 \}, B = \{ 2, 6 \}, C = \{ 4, 6 \}, and D = \{ 4, 6, 8 \}.  Determine \\which of these sets are subsets of which other of these sets.\\
\setlength\parindent{24pt}$ B \subset A$ and $ C \subset D$ and $C \subset A$\\

~\\

\setlength\parindent{0pt}10) Determine whether these statements are true or false. \\
\setlength\parindent{24pt}a) $\o \in \{ \o \}$ \\
\setlength\parindent{48pt} True.  The null set is an element of the set containing the null set. \\
\setlength\parindent{24pt}b) $\o \in \{ \o, \{ \o \}\} $\\
\setlength\parindent{48pt} True.  The null set is an element of the set containing the null set. \\
\setlength\parindent{24pt}c) $ \{ \o \} \in \{ \o \} $ \\ 
\setlength\parindent{48pt} False.  The set containing the null set is not an element in the set containing the null set. \\
\setlength\parindent{24pt}d) $\{ \o \} \in \{\{\o\}\}$ \\
\setlength\parindent{48pt} True.  The set containing the null set is an element of the set containing the set \\containing the null set. \\
\setlength\parindent{24pt}e) $\{ \o \} \subset \{ \o \{ \o \} \}$ \\
\setlength\parindent{48pt} True.  The set containing the null set is a strict subset of the set containing the null set and \\the set inside a set containing the null set. \\
\setlength\parindent{24pt}f) $\{\{\o\}\} \subset \{\o, \{\o\}\}$ \\
\setlength\parindent{48pt} True.  The set containing the null set is a subset of the set containing the set containing \\the null set.  Since there are two elements in this set on the right, this is a subset since the \\left is not equal to the right.\\
\setlength\parindent{24pt}g) $\{\{\o\}\} \subset \{\{\o\}, \{\o\}\}$ \\
\setlength\parindent{48pt} False.  This is interesting.  Since no items are repeated in a subset, the two null sets on the right are\\ treated as one.  Since this is a proper subset, the subset cannot be equal.  If the subset sign \\was $\subseteq$ then this would actually be true. \\



\end{flushleft}
\end{document}