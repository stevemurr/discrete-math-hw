\documentclass{article}
\usepackage[margin=1in]{geometry}
\usepackage{graphicx}
\usepackage[fleqn]{amsmath}
\usepackage{color}
\usepackage{lipsum}
\begin{document}
\setcounter{totalnumber}{5}
   \begin{flushright}
      \Large\textbf{Steven Murr}\\
      \large\textit{HW 2.1} \\
       \large\textit{Problems = \{1, 3, 5-11, 17, 19, 21, 23, 24, 27, 46 \}}
   \end{flushright}
\begin{flushleft}
\makeatletter% Set distance from top of page to first float
\setlength{\@fptop}{5pt}
\makeatother

\setlength\parindent{0pt}1) List the members of these sets.\\
\setlength\parindent{24pt}a) \{ $x | x$ is a real number such that $x^2 = 1$ \} \\
\setlength\parindent{48pt} \{ $\pm 1$ \} | 1 and -1 are the only real numbers that when squared equal 1. \\
\setlength\parindent{24pt}b) \{ $x | x$ is a positive integer less than 12 \} \\
\setlength\parindent{48pt} \{ 0, 1, 2, 3, 4, 5, 6, 7, 8, 9, 10, 11 \} \\
\setlength\parindent{24pt}c) \{ $x | x$ is the square of an integer and $x < 100$ \} \\
\setlength\parindent{48pt} \{ 0, 1, 4, 9, 16, 25, 36, 49, 64, 81 \} \\
\setlength\parindent{24pt}d) \{ $x | x$ is an integer such that $x^2 = 2$ \} \\
\setlength\parindent{48pt} \{ \o \} | Since no integer x when squared equals 2, we have the empty set or null set. \\

~\\
\setlength\parindent{0pt}3) For each of these pairs of sets, determine whether the first is a subset of the second, the second is a subset of the first, or neither is a subset of the other.\\
\setlength\parindent{24pt}a) The set of airline flights from New York to New Delhi, the set of nonstop airline flights from\\ New York to New Delhi. \\
\setlength\parindent{48pt} The set of nonstop airline flights is a subset of airline flights.  Airline flights could be a broad\\ representation of all types of flights between the two locations and nonstop constituting\\ a specific type of airline flight.\\
\setlength\parindent{24pt}b) The set of people who speak English, the set of people who speak Chinese. \\
\setlength\parindent{48pt} Neither is a subset of the other.  English and Chinese are both specific languages.  If English \\was "a language" or "any language" then a subset relationship could be derived.\\
\setlength\parindent{24pt}c) The set of flying squirrels, the set of living creatures that can fly.\\
\setlength\parindent{48pt} The set of flying squirrels is a subset of the living creatures that can fly. \\

~\\
\setlength\parindent{0pt}5) Determine whether each of these pairs of set are equals.\\
\setlength\parindent{24pt}a) \{ 1, 3, 3, 3, 5, 5, 5, 5 \}, \{ 5, 3, 1 \}
\setlength\parindent{48pt} These two sets are equals.  Two sets are equal if and \\only if they have the same elements.  In this case, 1, 3 and 5.\\
\setlength\parindent{24pt}b) \{\{ 1 \}\}, \{ 1, \{ 1\}\} \\
\setlength\parindent{48pt} These two sets are not equal however the first set is a subset of the second set.\\
\setlength\parindent{24pt}c) $\o$, $\{ \o \}$ \\
\setlength\parindent{48pt} These two sets are not equal because we are comparing the empty \\set and the empty set within the empty set.\\

~\\
\setlength\parindent{0pt}6) Suppose that A = \{ 2, 4, 6 \}, B = \{ 2, 6 \}, C = \{ 4, 6 \}, and D = \{ 4, 6, 8 \}.  Determine \\which of these sets are subsets of which other of these sets.\\
\setlength\parindent{24pt}$ B \subset A$ and $ C \subset D$ \\

~\\
\setlength\parindent{0pt}7) For each of the following sets, determine whether 2 is an element of that set. \\
\setlength\parindent{24pt}a) \{ $x \in \rm I\!R |  x$ is an integer greater than 1 \} \\
\setlength\parindent{48pt} 2 is an element of the set of all real numbers where x is an integer greater than 1. \\
\setlength\parindent{24pt}b) \{ $x \in \rm I\!R |  x$ is the square of an integer \} \\
\setlength\parindent{48pt} 2 is not an element of the set of reals where x is the square of an integer.  No integer \\squared equals 2. \\
\setlength\parindent{24pt}c) \{ 2, \{ 2 \}\} \\
\setlength\parindent{48pt} 2 is an element of the set.  However we only count the 2 immediately inside the set and \\not the set containing the set containing 2.\\
\setlength\parindent{24pt}d) \{\{ 2 \}, \{\{ 2 \}\}\} \\
\setlength\parindent{48pt} 2 is not an element of the set.  These are sets within a set. \\
\setlength\parindent{24pt}e) \{\{ 2 \}, \{ 2, \{ 2 \}\}\} \\
\setlength\parindent{48pt} 2 is not an element of the set.  Again these are sets within a set. \\
\setlength\parindent{24pt}f) \{\{\{ 2 \}\}\} \\
\setlength\parindent{48pt} 2 is not an element of the set.  This is a set within a set within a set containing 2. \\

~\\
\setlength\parindent{0pt}8) For each of the sets in Exercise 7, determine whether \{2\} is an element of that set.\\
\setlength\parindent{24pt}a) \{ $x \in \rm I\!R |  x$ is an integer greater than 1 \} \\
\setlength\parindent{48pt} The set containing 2 is not an element of the set. \\
\setlength\parindent{24pt}b) \{ $x \in \rm I\!R |  x$ is the square of an integer \} \\
\setlength\parindent{48pt} The set containing 2 is not an element of the set. \\
\setlength\parindent{24pt}c) \{ 2, \{ 2 \}\} \\
\setlength\parindent{48pt} The set containing 2 is an element of this set. \\
\setlength\parindent{24pt}d) \{\{ 2 \}, \{\{ 2 \}\}\} \\
\setlength\parindent{48pt} The set containing 2 is an element of this set. \\
\setlength\parindent{24pt}e) \{\{ 2 \}, \{ 2, \{ 2 \}\}\} \\
\setlength\parindent{48pt} The set containing 2 is an element of this set. \\
\setlength\parindent{24pt}f) \{\{\{ 2 \}\}\} \\
\setlength\parindent{48pt} The set containing 2 is not an element of this set. \\

~\\
\setlength\parindent{0pt}9) Determine whether each of these statements is true or false. \\
\setlength\parindent{24pt}a) $0 \in \{ \o \}$ \\
\setlength\parindent{48pt} False.  0 is not an element of the null set. \\
\setlength\parindent{24pt}b) $\o \in \{ \o \} $ \\
\setlength\parindent{48pt} False.  The null set is not an element of the set containing the null set. \\
\setlength\parindent{24pt}c) $\{ 0 \} \subset \o$ \\
\setlength\parindent{48pt} False.  The set containing 0 is not a subset of the null set. \\
\setlength\parindent{24pt}d) $\o \subset \{ 0 \}$ \\
\setlength\parindent{48pt} True.  The null set is considered to be a part of every other set, thus it is a subset of $ \{ 0 \} $ \\
\setlength\parindent{24pt}e) $ \{ 0 \} \in \{ 0 \}$ \\
\setlength\parindent{48pt} False.  The set containing 0 is not an element of the set containing 0. \\
\setlength\parindent{24pt}f) $\{ 0 \} \subset \{ 0 \}$ \\
\setlength\parindent{48pt} False.  The set containing 0 is not a proper subset of the set containing 0 because \\proper subsets don't allow equality.\\
\setlength\parindent{24pt}g) $ \{ \o \} \subseteq \{\o\}$ \\
\setlength\parindent{48pt} True.  The set containing the null set is a subset of the set containing the null set. \\

~\\
  c  

~\\
\setlength\parindent{0pt}11) Determine whether each of these statements is true or false. \\
\setlength\parindent{24pt}a) $x \in \{x\}$ \\
\setlength\parindent{48pt} True.  x is an element of x.\\
\setlength\parindent{24pt}b) $ \{x\} \subseteq \{x\}$ \\
\setlength\parindent{48pt} True.  \{ x \} is a subset of \{x\}.  This type of subset is true when sets are equal. \\
\setlength\parindent{24pt}c) $\{x\} \in \{x\}$ \\
\setlength\parindent{48pt} False.  The set containing x is not an element of the set containing x.  The left side must appear\\ exactly as it is inside of the set to be an element of the set.\\
\setlength\parindent{24pt}d) $\{x \} \in \{\{x\}\}$ \\ 
\setlength\parindent{48pt} True.  The set containing x is an element of the set containing the set containing x. \\
\setlength\parindent{24pt}e) $\o \subseteq \{x\}$ \\
\setlength\parindent{48pt} True.  The null set is a subset of every other set. \\
\setlength\parindent{24pt}f) $\o \in \{x\}$ \\
\setlength\parindent{48pt} False.  The null set is not an element of the set containing x. \\ 

~\\
\setlength\parindent{0pt}17) Suppose that A, B and C are sets such that $A \subseteq B$ and $B \subseteq C$.  Show that $A \subseteq C$. \\
\setlength\parindent{24pt} This uses an idea called extensionality.  Its a principle that refers to judging objects to be equal if \\they have the same external properties.  The idea is that given the same input then functions that \\look different will still yield the same output.  For some value $x \in A$ and since $A \subseteq B$ then $x \in B$ \\ and if $B \subseteq C$ then then $x \in C$ and if x is an element of A, B and C, then $A \subseteq C$. \\

~\\
\setlength\parindent{0pt}19) What is the cardinality of each of these sets? \\
\setlength\parindent{24pt}a) $\{ a \}$ \\
\setlength\parindent{48pt} Cardinality of 1. \\
\setlength\parindent{24pt}b) $\{\{a\}\}$\\
\setlength\parindent{48pt} Cardinality of 1. \\
\setlength\parindent{24pt}c) $\{ a, \{a\}\}$ \\
\setlength\parindent{48pt} Cardinality of 2. \\
\setlength\parindent{24pt}d) $\{a, \{a\}, \{a,\{a\}\}\}$ \\
\setlength\parindent{48pt} Cardinality of 3. \\

~\\
\setlength\parindent{0pt}21) Find the power set of each of these sets, where a and b are distinct elements. \\
\setlength\parindent{24pt}a) $\{a\}$ \\
\setlength\parindent{48pt} P(a) = \{\{ a \}, $\o$ \} \\
\setlength\parindent{24pt}b) $\{ a, b \}$ \\ 
\setlength\parindent{48pt} P(a, b)  = \{$\o$, \{a\}, \{b\}, \{a, b\}\} \\ 
\setlength\parindent{24pt}c) $\{\o, \{\o\}\}$ \\
\setlength\parindent{48pt} $\{ \o, \{ \o\}, \{\{\o\}\}\}, \{\o, \{\o\}\}$ \\

~\\
\setlength\parindent{0pt}23) How many elements does each of these sets have where $a$ and $b$ are distinct elements? \\
\setlength\parindent{24pt}a) $P(\{ a, b, \{ a, b \}\})$ \\
\setlength\parindent{48pt} The number of distinct elements is $2^c$ where c is the cardinality.  In this case it's $2^3 = 8$. \\
\setlength\parindent{24pt}b) $ P(\{\o, a, \{a\}, \{\{a\}\}\}) $ \\
\setlength\parindent{48pt} The number of distinct elements is $2^c$ where c is the cardinality.  In this case it's $2^4 = 16$. \\
\setlength\parindent{24pt}c) $P(P(\o))$ \\
\setlength\parindent{48pt} The number of distinct elements is $2^c$ where c is the cardinality.  In this case it's $2^1 = 2$. \\ 

~\\

\setlength\parindent{0pt}24) Determine whether each of theses sets is the power set of a set, where $a$ and $b$ are distinct elements. \\
\setlength\parindent{24pt}a) $\o$ \\
\setlength\parindent{48pt} No.  The power set of the empty set would be $\{\o\}$ \\
\setlength\parindent{24pt}b) $\{\o, \{a\}\}$ \\
\setlength\parindent{48pt} Yes.  This is a power set. \\
\setlength\parindent{24pt}c) $\{\o, \{a\}, \{\o,a\}\}$ \\
\setlength\parindent{48pt} No.  This is not a power set.  \\
\setlength\parindent{24pt}d) $\{\o, \{a\}, \{b\}, \{a, b\}\}$ \\
\setlength\parindent{48pt} Yes.  This is a power set.  \\

~\\

\setlength\parindent{0pt}27) Let $A = \{a, b, c, d \}$ and $B = \{y,z\}$ \\
\setlength\parindent{24pt}a) $A x B$ \\
\setlength\parindent{48pt} $A x B = \{(a, y), (a, z), (b, y), (b, z), (c, y), (c, z), (d, y), (d, z)\}$ \\
\setlength\parindent{24pt}b) $B x A$ \\
\setlength\parindent{48pt} $B x A = \{(y, a), (y, b), (y, c), (y, d), (z, a), (z, b), (z, c), (z, d)\}$ \\

~\\
\setlength\parindent{0pt}46) This exercise presents Russel's Paradox.  Let S be the set that contains a set x if the set x does not belong to itself, so that $S = \{x | x \notin x \}$ \\
\setlength\parindent{24pt}a) Show the assumption that S is a member of S leads to a contradiction. \\
\setlength\parindent{48pt} If S is a member of S then $S = \{x | x \in x \}$ would be true, however the definition of S \\states that $S = \{x | x \notin x \}$.  Therefore S would then contain elements of x and NOT\\ contain elements of x.\\
\setlength\parindent{24pt}b) Show the assumption that S is not a member of S leads to a contradiction.\\
\setlength\parindent{48pt} If S is defined as the set x not containing itself, then if S is not a member of S, it \\should contain itself because S is not a member of itself.

\end{flushleft}
\end{document}