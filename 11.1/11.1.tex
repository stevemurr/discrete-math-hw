\documentclass{article}
\usepackage[margin=1in]{geometry}
\usepackage{graphicx}
\usepackage[fleqn]{amsmath}
\usepackage{color}
\usepackage{lipsum}
\begin{document}
\setcounter{totalnumber}{5}
   \begin{flushright}
      \Large\textbf{Steven Murr}\\
      \large\textit{HW 11.1} \\
      \large\textit{ Problems = \{ 1 - 9 odd, 11a, 44\}}
   \end{flushright}
\begin{flushleft}
\makeatletter% Set distance from top of page to first float
\setlength{\@fptop}{5pt}
\makeatother

\setlength\parindent{0pt}1) Which of these graphs are trees? \\
**See attached paper. \\
~\\
\setlength\parindent{0pt}3) Answer these questions about the rooted tree illustrated. \\
\setlength\parindent{24pt}a) Which vertex is the root? \\
\setlength\parindent{48pt} a \\
\setlength\parindent{24pt}b) Which vertices are internal? \\
\setlength\parindent{48pt} a, b, c, d, f, h, j, q, t \\
\setlength\parindent{24pt}c) Which vertices are leaves? \\
\setlength\parindent{48pt} e, g, i, l, m, n, o, p, r, s, u \\
\setlength\parindent{24pt}d) Which vertices are children of j? \\
\setlength\parindent{48pt} q,r \\
\setlength\parindent{24pt}e) Which vertex is the parent of h? \\
\setlength\parindent{48pt} c \\
\setlength\parindent{24pt}f) Which vertices are siblings of o? \\
\setlength\parindent{48pt} p \\
\setlength\parindent{24pt}g) Which vertices are ancestors of m? \\
\setlength\parindent{48pt} f, b, a \\
\setlength\parindent{24pt}h) Which vertices are descendants of b? \\
\setlength\parindent{48pt} e, f, l, m, n \\
~\\
\setlength\parindent{0pt}5) Is the rooted tree in Exercise 3 a full m-ary tree for some positive integer m? \\
\setlength\parindent{24pt} It is not a full m-ary tree because some internal vertices have 1, 2 or 3 children.  A full m-ary tree requires no more than m children on every vertex. \\
~\\
\setlength\parindent{0pt}7) What is the level of each vertex of the rooted tree in Exercise 3? \\
\setlength\parindent{24pt} Level 0: a - Level 1: b,c,d - Level 2: e,f,g,h,i,j,k - Level 3: l,m,n,o,p,q,e - Level 4: s,t - Level 5: u \\
~\\
\setlength\parindent{0pt} Draw the subtree of the tree in Exercise 3 that is rooted at: \\
\setlength\parindent{24pt} a \\
\setlength\parindent{24pt} c \\
\setlength\parindent{24pt} e \\
**See attached paper. \\
~\\
\setlength\parindent{0pt}11a) How many nonisomorphic unrooted trees are there with three vertices? \\
\setlength\parindent{24pt} 1 \\
~\\
\setlength\parindent{0pt}44) Show that every tree can be colored using two colors.  The rooted Fibonacci trees $T_n$ are defined recursively in the following way.  $T_1$ and $T_2$ are both the rooted tree consisting of a single vertex, and for n = 3,4 . . . the rooted tree $T_n$ is constructed from a root with $T_{n-1}$ as its left subtree and $T_{n-2}$ as its right subtree. \\
**See attached paper.

\end{flushleft}
\end{document}