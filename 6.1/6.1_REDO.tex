\documentclass{article}
\usepackage[margin=1in]{geometry}
\usepackage{graphicx}
\usepackage[fleqn]{amsmath}
\usepackage{color}
\usepackage{lipsum}
\begin{document}
\setcounter{totalnumber}{5}
   \begin{flushright}
      \Large\textbf{Steven Murr}\\
      \large\textit{REDO 6.1} \\
   \end{flushright}
\begin{flushleft}
\makeatletter% Set distance from top of page to first float
\setlength{\@fptop}{5pt}
\makeatother

\setlength\parindent{0pt}28) How many license plates can be made using either three digits followed by three uppercase English letters or three uppercase English letters followed by three digits.  \\
\setlength\parindent{24pt}This has the following properties: DDD EEE or EEE DDD, repeats ok, D = 10, E = uppercase \\English letter = 26. \\
\setlength\parindent{24pt}This appears to be a product rule combined with the sum rule.  $10^3 + 26^3 = 2(18576) = 37152$.  We multiply by two because the two orders are both $10^3 + 26^3$\\
~\\


\end{flushleft}
\end{document}