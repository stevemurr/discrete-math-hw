
\documentclass{article}
\usepackage[margin=1in]{geometry}
\usepackage[fleqn]{amsmath}
\begin{document}
   \begin{flushright}
      \Large\textbf{Steven Murr}\\
      \large\textit{HW 1.1}
   \end{flushright}
\begin{flushleft}
1) Which of these sentences are propositions? What are the truth values of those that are propositions?\newline
\\\setlength\parindent{24pt}a) Boston is the capital of Massachusetts \\
\setlength\parindent{48pt} This is a proposition and it is True \\
\setlength\parindent{0pt}
\setlength\parindent{24pt}b) Miami is the capital of Florida. \\
\setlength\parindent{48pt} This is a proposition and it is False \\
\setlength\parindent{0pt} 
\setlength\parindent{24pt}c) 2 + 3 = 5 \\
\setlength\parindent{48pt} This is a proposition and it is True \\
\setlength\parindent{0pt}
\setlength\parindent{24pt}d) 5 + 7 = 10 \\
\setlength\parindent{48pt} This is a proposition and it is False \\
\setlength\parindent{0pt}
\setlength\parindent{24pt}e) x + 2 = 11 \\ 
\setlength\parindent{48pt} This is not a proposition \\ 
\setlength\parindent{0pt}
\setlength\parindent{24pt}f) Answer this question \\
\setlength\parindent{48pt} This is not a proposition \\ 
\setlength\parindent{0pt}

~\\3) What is the negation of each of these propositions? \\ 
~\\\setlength\parindent{24pt}a) Mei has an MP3 player. \\
\setlength\parindent{48pt} Mei does not have an MP3 player. \\
\setlength\parindent{24pt}b) There is no pollution in New Jersey.\\
\setlength\parindent{48pt} There is pollution in New Jersey.\\
\setlength\parindent{24pt}c) 2 + 1 = 3\\
\setlength\parindent{48pt} 2 + 1 $\neq$ 3 \\
\setlength\parindent{24pt}d) The summer in Maine is hot and sunny. \\
\setlength\parindent{48pt} The summer in Maine is not hot nor sunny.\\
~\\
\setlength\parindent{0pt}
13) Let p and q be the propositions \\
\setlength\parindent{24pt} p : You drive over 65 miles per hour.\\
\setlength\parindent{24pt} q : You get a speeding ticket. \\
~\\Write these propositions using p and q and logical connectives\\
\setlength\parindent{24pt} (including negations).\\
a) You do not drive over 65 miles per hour.\\
\setlength\parindent{48pt} $\neg p$ \\ 
\setlength\parindent{24pt}b) You drive over 65 miles per hour, but you do not get a speeding ticket.\\
\setlength\parindent{48pt}$p \land\neg q$\\
\setlength\parindent{24pt}c) You will get a speeding ticket if you drive over 65 miles per hour. \\
\setlength\parindent{48pt}$p \rightarrow q$ \\
\setlength\parindent{24pt}d) If you do not drive over 65 miles per hour, then you will not \\
\setlength\parindent{24pt}get a speeding ticket.\\
\setlength\parindent{48pt}$\neg p \rightarrow \neg q$\\

~\\\setlength\parindent{0pt}17) Determine whether each of these conditional statements is true or false.\\
\setlength\parindent{24pt}a) If 1 + 1 = 2, then 2 + 2 = 5.\\
\setlength\parindent{48pt} True to False -- "Broken Promise" -- The statement is False\\
\setlength\parindent{24pt}b) If 1 + 1 = 3, then 2 + 2 = 4.\\
\setlength\parindent{48pt} False to True -- The statement is True.\\
\setlength\parindent{24pt}c) If 1 + 1 = 3, then 2 + 2 = 5. \\
\setlength\parindent{48pt} False to False -- The statement is True.\\
\setlength\parindent{24pt}d) If monkeys can fly, then 1 + 1 = 3. \\ 
\setlength\parindent{48pt}False to False = The statement is True. \\  
~\\
\setlength\parindent{0pt}20) For each of these sentences, determine whether an inclusive or, or an exclusive or, is intended.  Explain your answer.\\
\setlength\parindent{24pt}a) Experience with C++ or Java is required.\\
\setlength\parindent{48pt} This is an inclusive or.  You will not be disqualified from getting \\
\setlength\parindent{48pt}this job if you know both languages.\\
\setlength\parindent{24pt}b) Lunch includes soup or salad.\\
\setlength\parindent{48pt}This is an exclusive or.  If your lunch included both, it would be \\
\setlength\parindent{48pt}stated as "Lunch includes soup AND salad."\\
\setlength\parindent{24pt}c) To enter the country you need a passport or a voter registration card.\\
\setlength\parindent{48pt} Inclusive or.  You will not be disqualified if you have both.\\
\setlength\parindent{24pt}d) Publish or perish.\\
\setlength\parindent{48pt} Exclusive or.  The statement infers that you either publish a book \\
\setlength\parindent{48pt}or are forgotten.\\
~\\\setlength\parindent{0pt}27) State the converse, contrapositive, and inverse of each of these conditional statements.\\
\setlength\parindent{24pt}a) If it snows today, I will ski tomorrow.\\
\setlength\parindent{48pt}Inverse: If it does not snow today, I will not ski tomorrow.\\
\setlength\parindent{48pt}Converse: I will ski tomorrow only if it snows today.\\
\setlength\parindent{48pt}Contrapositive: If I don't ski tomorrow, then it won't snow today.\\
\setlength\parindent{24pt}b) I come to class whenever there is going to be a quiz.\\
\setlength\parindent{48pt}Inverse: If there is not going to be a quiz, then I don't come to class.\\
\setlength\parindent{48pt}Converse: If I come to class then there will be a quiz.\\
\setlength\parindent{48pt}Contrapositive: If I don't come to class, then there isn't going to be a quiz.\\
\setlength\parindent{24pt}c) A positive integer is a prime only if it has no divisors other than 1 and itself.\\
\setlength\parindent{48pt}Inverse: If a positive integer is not a prime then it has divisors other than 1 and itself.\\
\setlength\parindent{48pt}Converse: If a positive integer has no divisors other than 1 and itself then it is a prime.\\
\setlength\parindent{48pt}Contrapositive: If a positive integer has a divisor other than 1 and itself, then it is not prime..\\
\setlength\parindent{0pt}31) Construct a truth table for each of these compound propositions.\\
\setlength\parindent{24pt}a) $p \land \neg p$\\
\setlength\parindent{24pt}~\\
\begin{tabular} {| l | c || r |}
  \hline
  $p$ & $\neg p$ & $p \land \neg p$ \\ \hline
  T & F & F \\ \hline
  F & T & F \\ \hline
\end{tabular}\\
~\\\setlength\parindent{24pt}b) $p \lor \neg p$\\
~\\\begin{tabular} {| l | c || r |}
  \hline
  $p$ & $\neg p$ & $p$ $\lor \neg p$ \\ \hline
  T & F & T \\ \hline
  F & T & T \\ \hline
\end{tabular} \\
~\\\setlength\parindent{24pt}c) $(p \lor \neg q) \rightarrow q$ \\
\begin{displaymath}
\begin{array} {| c | c | c | c | c }
  $$p$$
  & $q$
  & $$\neg q$$
  & $$(p \lor \neg q)$$
  & $$(p \lor \neg q) \rightarrow q$$ \\ \hline
T & T & F & T & T \\
T & F & T & T & F \\
F & T & F & F & T \\
F & F & T & T & F \\
\end{array}
\end{displaymath}
~\\\setlength\parindent{24pt}d) $(p \land q) \rightarrow (p \lor q)$
\begin{displaymath}
\begin{array} {| c | c | c | c | c }
  $$p$$
  & $$q$$
  & $$(p \land q)$$
  & $$(p \lor q)$$
  & $$(p \land q) \rightarrow (p \lor q)$$ \\ \hline
T & T & T & T & T \\
T & F & F & T & T \\
F & T & F & T & T \\
F & F & F & F & T \\
\end{array}
\end{displaymath}
~\\\setlength\parindent{24pt}e) $(q \rightarrow \neg p) \iff (\neg q \rightarrow \neg p)$ \\
\begin{displaymath}
\begin{array} {| c | c | c | c | c | c | c}
  $$p$$
  & $$q$$
  & $$\neg p$$
  & $$\neg q$$
  & $$p \rightarrow q$$
  & $$\neg q \rightarrow \neg p$$
  & $$(p \rightarrow q) \iff (\neg q \rightarrow \neg p)$$ \\ \hline
T & T & F & F & T & T & T \\
T & F & F & T & F & F & T \\
F & T & T & F & T & T & T \\
F & F & T & T & T & T & T \\
\end{array}
\end{displaymath}
~\\\setlength\parindent{24pt}f) $(p \rightarrow q) \rightarrow (q \rightarrow p)$  \\
\begin{displaymath}
\begin{array} {| c | c | c | c | c | c }
  $$p$$
  & $$q$$
  & $$p \rightarrow q$$
  & $$q \rightarrow p$$
  & $$(p \rightarrow q) \rightarrow (q \rightarrow p)$$ \\ \hline
T & T & T & T & T \\
T & F & F & T & T \\
F & T & T & F & F \\
F & F & T & T & T \\
\end{array}
\end{displaymath}

~\\\setlength\parindent{0pt}32) Construct a truth table for each of these compound propositions.
~\\\setlength\parindent{24pt}a) $p \rightarrow \neg p$
\begin{displaymath}
\begin{array}{| c | c | c}
$$ p $$
& $$ \neg p $$
& $$ p \rightarrow \neg p $$ \\ \hline
T & F & F \\
F & T & T \\
\end{array}
\end{displaymath}

~\\\setlength\parindent{24pt}b) $p \iff \neg p$
\begin{displaymath}
\begin{array}{| c | c | c}
$$ p $$
& $$ \neg p $$
& $$ p \iff \neg p $$ \\ \hline
T & F & F \\
F & T & F \\
\end{array}
\end{displaymath}

~\\\setlength\parindent{24pt}c) $p \oplus (p \lor q)$
\begin{displaymath}
\begin{array}{| c | c | c | c}
$$p$$
& $$ q $$
& $$ (p \lor q)$$
& $$ p \oplus (p \lor q)$$ \\ \hline
T & T & T & F \\
T & F & T & F \\
F & T & T & T \\
F & F & F & F \\
\end{array}
\end{displaymath}

~\\\setlength\parindent{24pt}d) $(p \land q) \rightarrow (p \lor q)$ \\
\begin{displaymath}
\begin{array}{| c | c | c | c | c}
$$p$$
& $$q$$
& $$(p \land q)$$
& $$(p \lor q)$$
& $$(p \land q) \rightarrow (p \lor q)$$ \\ \hline
T & T & T & T & T \\ 
T & F & F & T & T \\
F & T & F & T & T \\
F & F & F & F & T \\
\end{array}
\end{displaymath}

~\\\setlength\parindent{24pt}e) $(q \rightarrow \neg p) \iff (p \iff q)$ \\ 
\begin{displaymath}
\begin{array}{| c | c | c | c | c | c}
$$p $$
&$$ q $$
&$$\neg p$$
&$$(q \rightarrow \neg p)$$
&$$(p \iff q)$$
&$$(q \rightarrow \neg p) \iff (p \iff q)$$ \\ \hline

T & T & F & F & T & F \\
T & F & F & T & F & F \\
F & T & T & T & F & F \\
F & F & T & T & T & T \\

\end{array}
\end{displaymath}
~\\~\\\setlength\parindent{24pt}f) $(p \iff q) \oplus (p \iff \neg q)$ \\ 

\begin{displaymath}
\begin{array}{| c | c | c | c | c | c}
$$p $$
&$$ q $$
&$$\neg q$$
&$$(p \iff q)$$
&$$(p \iff \neg q)$$
&$$(p \iff q) \oplus (p \iff \neg q)$$ \\ \hline

T & T & F & T & F & T \\
T & F & T & F & T & T \\
F & T & F & F & T & T \\
F & F & T & T & F & T \\

\end{array}
\end{displaymath}

~\\\setlength\parindent{0pt}36) Construct a truth table for each of these compound propositions.
~\\~\\\setlength\parindent{24pt}a) $(p \lor q) \lor r$ 
\begin{displaymath}
\begin{array}{| c | c | c | c | c}
$$p$$
& $$q$$
& $$r$$
& $$(p \lor q)$$
& $$(p \lor q) \lor r$$ \\ \hline
T & T & T & T & T \\ 
T & T & F & T & T \\
T & F & T & T & T \\
T & F & F & T & T \\
F & T & T & T & T \\ 
F & T & F & T & T \\
F & F & T & F & T \\
F & F & F & F & F \\
\end{array}
\end{displaymath}
~\\~\\\setlength\parindent{24pt}b) $(p \lor q) \land r$ \\ 
\begin{displaymath}
\begin{array}{| c | c | c | c | c}
$$p$$
& $$q$$
& $$r$$
& $$(p \lor q)$$
& $$(p \lor q) \land r$$ \\ \hline
T & T & T & T & T \\ 
T & T & F & T & F \\
T & F & T & T & T \\
T & F & F & T & F \\
F & T & T & T & T \\ 
F & T & F & T & F \\
F & F & T & F & F \\
F & F & F & F & F \\
\end{array}
\end{displaymath}

~\\~\\\setlength\parindent{24pt}c) $(p \land q) \lor r$ \\ 
\begin{displaymath}
\begin{array}{| c | c | c | c | c}
$$p$$
& $$q$$
& $$r$$
& $$(p \land q)$$
& $$(p \land q) \lor r$$ \\ \hline
T & T & T & T & T \\ 
T & T & F & T & T \\
T & F & T & F & T \\
T & F & F & F & F \\
F & T & T & F & T \\ 
F & T & F & F & F \\
F & F & T & F & T \\
F & F & F & F & F \\
\end{array}
\end{displaymath}

~\\~\\~\\\setlength\parindent{24pt}d) $(p \land q) \land r$ \\ 
\begin{displaymath}
\begin{array}{| c | c | c | c | c}
$$p$$
& $$q$$
& $$r$$
& $$(p \land q)$$
& $$(p \land q) \land r$$ \\ \hline
T & T & T & T & T \\ 
T & T & F & T & F \\
T & F & T & F & F \\
T & F & F & F & F \\
F & T & T & F & F \\ 
F & T & F & F & F \\
F & F & T & F & F \\
F & F & F & F & F \\
\end{array}
\end{displaymath}

~\\~\\\setlength\parindent{24pt}e) $(p \lor q) \land \neg r$ \\ 
\begin{displaymath}
\begin{array}{| c | c | c | c | c | c}
$$p$$
& $$q$$
& $$r$$
& $$\neg r$$
& $$(p \lor q)$$
& $$(p \lor q) \land \neg r$$ \\ \hline
T & T & T & F & T & F \\ 
T & T & F & T & T & T \\
T & F & T & F & T & F \\
T & F & F & T & T & T \\
F & T & T & F & T & F \\ 
F & T & F & T & T & T \\
F & F & T & F & F & F \\
F & F & F & T & F & F \\
\end{array}
\end{displaymath}

~\\~\\\setlength\parindent{24pt}f) $(p \land q) \lor \neg r$ \\ 
\begin{displaymath}
\begin{array}{| c | c | c | c | c | c}
$$p$$
& $$q$$
& $$r$$
& $$\neg r$$
& $$(p \land q)$$
& $$(p \land q) \lor \neg r$$ \\ \hline
T & T & T & F & T & T \\ 
T & T & F & T & T & T \\
T & F & T & F & F & F \\
T & F & F & T & F & T \\
F & T & T & F & F & F \\ 
F & T & F & T & F & T \\
F & F & T & F & F & F \\
F & F & F & T & F & T \\
\end{array}
\end{displaymath}

~\\\setlength\parindent{0pt}45) The truth value of the negation of a proposition in fuzzy logic is 1 minus the truth value of the proposition.  What are the truth values of the statements "Fred is not happy" and "John is not happy"?\\

~\\"Fred is happy" has a truth value of 0.8.  
The Truth value for "Fred is not happy" is 0.2.  1 - 0.8 = 0.2.\\

~\\"John is happy" has a truth value of 0.4.   \\
The truth value of "John is not happy" is 0.6.  1 - 0.4 = 0.6.\\

~\\\setlength\parindent{0pt}46) The truth value of the conjunction of two propositions in fuzzy logic is the minimum of the truth values of the two propositions.  What are the truth values of the statements "Fred and John are happy" and "Neither Fred nor John is happy?"\\

The truth value for "Fred and John are happy" is -- 0.4. \\
0.4 $<$ 0.8.  The truth value of the conjunction of two propositions in fuzzy logic is the minimum of the truth values of the two propositions. \\ 
~\\The truth value for the statement "Neither Fred nor John is happy?" is 0.2.  0.2 $<$ 0.6.

~\\\setlength\parindent{0pt}47) The truth value of the disjunction of two propositions in fuzzy logic is the maximum of the truth values of the two propositions.  What are the truth values of the statements "Fred is happy, or John is happy" and "Fred is not happy, or John is not happy?"\\
~\\
The truth value of "Fred is happy, or John is happy" is 0.8.  0.8 $>$ 0.6.\\
~\\The truth value of Fred is not happy, or John is not happy?" is 0.6.  0.6 $>$ 0.2.

~\\\setlength\parindent{0pt}48) Is the assertion "This statement is false" a proposition? \\ 
Yes, because the statement itself can be True or False.  Perhaps the person is lying, or misinformed what they made the assertion.\\

\end{flushleft}
\end{document}