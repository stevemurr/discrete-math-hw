\documentclass{article}
\usepackage[margin=1in]{geometry}
\usepackage{graphicx}
\usepackage[fleqn]{amsmath}
\usepackage{color}
\usepackage{lipsum}
\begin{document}
\setcounter{totalnumber}{5}
   \begin{flushright}
      \Large\textbf{Steven Murr}\\
      \large\textit{HW 8.2} \\
      \large\textit{ Problems = \{ 3abc, 4ab, 5, 8\} } 
   \end{flushright}
\begin{flushleft}
\makeatletter% Set distance from top of page to first float
\setlength{\@fptop}{5pt}
\makeatother

\setlength\parindent{0pt}3) Solve these recurrence relations together with the initial conditions given. \\
~\\ 
\setlength\parindent{24pt}a) $a_n = 2a_{n-1} $ for $n \geq 1, a_0 = 3$ \\
$a_n = 2a_{n-1} = a2^0 = 3$ \\
$a\cdot 1 = 3$ therefore $a = 3$ \\ 
3 is our constant and 2 is our root therefore $3 \cdot 2^n$ \\
~\\
b) $a_n = a_{n-1}$ for $ n \geq 1, a_0 = 2$ \\
$a_n = a_{n-1} = a\cdot 2^0 = a\cdot 1 = 2$ therefore $a_n = 2$ \\
~\\
c) $a_n = 5a_{n-1} - 6a{n-2}$ for $n \geq 2 when a_0 = 1$ and $a_1 = 0$ \\
$ r^2 - 5r + 6 $ factors as $(r-3)(r-2)$.  3 and 2 are our roots. \\
$a_0 = 1 = a\cdot2^0 - b\cdot3^0 $ which yields: \\
$a - b = 1$ \\ 
$a_1 = a\cdot2^1 - b\cdot3^1 = a\cdot2^n - b\cdot3^n = 0$ \\
~\\
We can see in the first equation that a is 3 and b is 2 however we can also solve by \\multiplying a-b = 1 by -2 to cancel out a and solve for b then plug b into the a-b equation. \\
~\\
\setlength\parindent{0pt}4) Solve these recurrence relations together with the initial conditions given.  \\
~\\
a) $a_n = a_{n-1} + 6a_{n-2} = r^2 - r - 6$ the factors of which are $(r-2)(r-3)$.  3 and 2 are our roots. \\
Thus we have $a\cdot 3^n + b\cdot 2^n$ \\ 
$a_0 = 3 = a\cdot 3^0 + b \cdot 2^0 = a + b = 3$ \\
$a_1 = 6 = a\cdot 3^1 + b\cdot 2^1 = 3a + 2b = 6$ multiply the top by -2 to cancel b\\
$a = 0$ we then plug 0 into the a+b equation and solve for b. \\ 
Consequently we discover that b = 3. \\
The remaining equation is: $3^n + 3\cdot 2^n$ \\
~\\
b) $a_n = 7a_{n-1} - 10a_{n-2}$ for $n \geq 2, a_0 = 2, a_1 = 1$ \\
$ r^2 - 7r + 10 $ which is $(r-5)(r-2)$ we convert to quadratic and factor \\ 
Our roots are 5 and 2 leading us to: \\
$a \cdot 5^n - b \cdot 2^n$ \\
$a_0 = 2 = a\cdot 5^0 - b\cdot 2^0 = a - b = 2$ \\
$a_1 = 1 = a\cdot 5^1 - b\cdot 2^1 = 5a - 2b = 1$ we multiply the top by -2 to cancel b and solve
\begin{align*}
-4 &= -2a + 2b \\
1 &= 5a - 2b \\
-3 &= 3a - divide \\
-1 &= a
\end{align*}
Plug a = -1 back into the a - b = 2 equation \\ 
-1 - b = 2 \\
b = -3 \\
Thus our equation is: $a_n = -1 \cdot 5^n - -3\cdot 2^n$ \\
~\\
\setlength\parindent{0pt}5) How many different messages can be transmitted in n microseconds using the two signals described in Exercise 19 in Section 8.1? \\
~\\
19) Messages are transmitted over a communications channel using two signals.  The transmittal of one signal requires 1 microsecond and the transmittal of the other signal requires 2 microseconds.  \\
a) Find a recurrence relation for the number of different messages consisting of sequences of these two signals, where each signal in the message is immediately followed by the next signal, that can be sent in n microseconds. \\ 
~\\
This initially felt fibonacci'ish in that you have a series in which messages immediately follow one another, therefore finding a message sent at the nth microsecond would be a matter of plugging n into the fibonacci formula.  Looking in the back of the book confirmed my suspicions with: \\
$a_n = \frac{1}{\sqrt{5}}(\frac{1+\sqrt{5}}{2})^n+1 - \frac{1}{\sqrt{5}}(\frac{1+\sqrt{5}}{2})^n+1$ \\
~\\
\setlength\parindent{0pt}8) A model for the number of lobsters caught per year is based on the assumption that the number of lobsters caught in a year is the average of the number caught in the two previous years. \\
a) Find a recurrence relation for $\{ L_n \}$, where $L_n$ is the number of lobsters caught in year n, under the assumption for this model.  \\
$L_n = \frac{L_{n-1} + L_{n-2}}{2}$ \\
~\\
b) Find $L_n$ if $100,000$ lobsters were caught in year 1 and 300,000 were caught in year 2. \\
Using the previous equation: $L_n = \frac{1}{2}L_{n-1} + \frac{1}{2}L_{n-2}$ \\
$r^2 -\frac{1}{2}r - \frac{1}{2}$ factor \\
$\frac{1}{2} (2r+1)(r-1)$ our roots are 1 and - $\frac{1}{2}$\\
Solving as before we get the following system of equations: \\
$-(\frac{1}{2})a_1 + a_2 = 100,000$ \\
$\frac{1}{4} a_1 + 1_2  = 300,000$ \\
If we solve the system as before we find that $k_1 = \frac{800000}{3}$ and $k_2 = \frac{700000}{3}$ with our general equation being $\frac{800000}{3}-(\frac{1}{2})^n + \frac{700000}{3}$\\
c) Find $L_n$ if 100,000 lobsters were caught in year 1 and 300,000 were caught in year 2. \\
~\\
$\frac{800000}{3}-(\frac{1}{2})^2 + \frac{700000}{3} = 166666.66 $ caught.

\end{flushleft}
\end{document}