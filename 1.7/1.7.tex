\documentclass{article}
\usepackage[margin=1in]{geometry}
\usepackage{graphicx}
\usepackage[fleqn]{amsmath}
\usepackage{color}
\usepackage{lipsum}
\begin{document}
\setcounter{totalnumber}{5}
   \begin{flushright}
      \Large\textbf{Steven Murr}\\
      \large\textit{HW 1.7}
   \end{flushright}
\begin{flushleft}
\makeatletter% Set distance from top of page to first float
\setlength{\@fptop}{5pt}
\makeatother
\setlength\parindent{0pt}1) Use a direct proof to show that the sum of two odd integers is even.\\
\setlength\parindent{24pt} We will use a direct proof to show that the sum of two odd integers, $m$ and $n$ is even.\\
\setlength\parindent{24pt} We know that 2k+1, where k is an integer will always yield an odd number.  Also we know that 2k \\
where k is an integer will always yield a positive number.\\
~\\\setlength\parindent{24pt}$n + m = 2k+1 + 2j+1$ - Write n + m as being the sum of the general form of and odd integers.\\
\setlength\parindent{24pt}$n + m = 2k + 2j + 2$ - Combine any like terms and look to factor.\\
\setlength\parindent{24pt}$n + m = 2(k + j + 1)$ - Factor out a two.  The sum of integers k + j + 1 is an integer due to the \\
rule that states that the sum of integers is an integer.\\
\setlength\parindent{24pt}$n + m = 2($integer$)$ - Thus the sum of two odd integers is always an even integer.\\
~\\
\setlength\parindent{0pt}2) Use a direct proof to show that the sum of two even integers is even.
\setlength\parindent{24pt} We will use a direct proof to show that the sum of two even integers, $m$ and $n$ is even.\\
\setlength\parindent{24pt} We know that 2k+1, where k is an integer will always yield an odd number.  Also we know that 2k \\
where k is an integer will always yield a positive number.\\
~\\
\setlength\parindent{24pt}$n + m = 2k + 2j$ - We make the sum of n and m equal to the sum of the general form of an even \\
integer represented by k and j. \\
\setlength\parindent{24pt}$n + m = 2(k + j)$ - Factor out a 2.  Since the sum of two integers is an integer we can treat k+j \\
as an integer.\\
\setlength\parindent{24pt}$n + m = 2($integer$)$ - Since 2 multiplied by an integer will always yield an even integer as stated \\
by the rule that for any integer k, that 2k will be an even integer, we can state that the \\
sum of two even integers will always yield an even integer.\\
~\\
\setlength\parindent{0pt}3) Show that the square of an even number is an even number using a direct proof.\\
\setlength\parindent{24pt} We will use a direct proof to show that the square of an even integer m is an even integer.\\
\setlength\parindent{24pt} We know that 2k+1, where k is an integer will always yield an odd number.  Also we know that 2k \\
where k is an integer will always yield a positive number.\\
~\\

\setlength\parindent{24pt}$m^2 = (2k)(2k) = 4k^2$ - A number squared is simply itself multiplied by itself.  The \\
products of integers lets us multiply two integers together to yield an integer.\\
~\\\setlength\parindent{24pt}$\sqrt{m^2} = \sqrt{4k^2}$ - We take the square root of both sides.\\
\setlength\parindent{24pt}$m = 2k$ - For every integer k, 2k will yield an even integer therefore, the square of an even\\
 number is an even number.\\

~\\
\setlength\parindent{0pt}5) Prove that if $m + n$ and $n + p$ are even integers, where m, n, and p are integers, then $m + p$ is even.  What kind of proof did you use?\\
\setlength\parindent{24pt} We will use a direct proof to show that the sum of integers m + p is even if m + n are even \\
and n + p are even. \\
\setlength\parindent{24pt} We know that for any integer k, that 2k will yield an even number.\\
\setlength\parindent{24pt} $m + n = 2p$ and $n + p = 2q$ - Set the sum of m + n equal to 2p and n + p equal to 2q. \\
\setlength\parindent{24pt} $m + p + n + n = m + p + 2n$ - Combine the n's to yield 2n\\
\setlength\parindent{24pt} $m + p + 2n = 2p + 2q$ - Now we subtract the 2n to get m + p alone on the left. \\
\setlength\parindent{24pt} $m + p = 2(p + q -n)$ - We factor out the 2.  The sums of integers tells us that the integers p + q - n \\
when added together yield an integer. \\
\setlength\parindent{24pt} $m + p = 2($integer$)$ - The general form of 2k has been achieved, thus we know that m + p is even. \\


\setlength\parindent{0pt}6) Use a direct proof to show that the product of two odd numbers is odd. \\
\setlength\parindent{24pt} We will use a direct proof to show that when multiplying to odd numbers together\\
 then the product is odd. \\
\setlength\parindent{24pt} We know that for any integer k, the form 2k + 1 will yield an odd number. \\
\setlength\parindent{24pt} $mp = (2k + 1)(2j + 1)$ - Set mp equal to the product of 2k+1 and 2j+1. \\
\setlength\parindent{24pt} $mp = 4kj + 2k + 2j +1$ - Multiply out the right side. \\
\setlength\parindent{24pt} $mp = 2(2kj + k + j) + 1$ - We then factor out the 2.  We also know the kj yields an integer due\\
 to products of integers.  We then use the sum of integers to simplify kj + k + j into the idea \\
 that it yields an integer. \\ 
\setlength\parindent{24pt} $mp = 2(integer) + 1$ - This simplified form shows us that the product of m and p will \\
be an odd number.\\
~\\
\setlength\parindent{0pt}10) Use a direct proof to show that the product of two rational numbers is rational. \\
\setlength\parindent{24pt} We will use a direct proof to show that the product of two rational numbers p and q, is rational.\\
\setlength\parindent{24pt} A rational number is any number that can be represented as a quotient of two integers.  \\
\setlength\parindent{24pt} If p = $\frac{r}{s}$ and q = $\frac{t}{v}$ we can then show: \\
\setlength\parindent{24pt}$pq = (\frac{r}{s})(\frac{t}{v}) = \frac{rt}{sv}$ \\
\setlength\parindent{24pt} Product of integers says the product of two integers is an integer.  Thus, $\frac{rt}{sv}$ represents a rational\\
 number expressed as a quotient of integers.\\
~\\
\setlength\parindent{0pt}11) Prove or disprove that the product of two irrational numbers is irrational. \\
\setlength\parindent{24pt} We will disprove that the product of two irrational numbers m and p, is irrational by \\
CounterExample.\\
\setlength\parindent{24pt} We will state that $\sqrt{2}$ is an irrational number.  An irrational number is a number that cannot \\
be represented as a ratio of integers. \\
\setlength\parindent{24pt} If $pq = (\sqrt{2})(\sqrt{2}) = 2$  | 2 is a rational number and can be represented as $\frac{2}{1}$.  Thus \\
the product of two irrational numbers is not always irrational. \\

\setlength\parindent{0pt}14) Prove that if $x$ is rational and $x \neq 0$, then $\frac{1}{x}$ is rational. \\
\setlength\parindent{24pt} A rational number is a number that can be expressed as a ratio of integers. \\
\setlength\parindent{24pt} Also, $\frac{1}{\frac{1}{x}}$ is equivalent to the numerator multiplied by the reciprocal ($\frac{1}{1}$)($\frac{x}{1}$) \\
\setlength\parindent{24pt} which is $\frac{x}{1}$.  The product of integers is always an integer so $\frac{x}{1}$ is a rational number since the \\
numerator and denominator are still integers. \\
~\\
\setlength\parindent{0pt}15) Use a proof by contraposition to show that if $x + y \geq 2$, where x and y are real \\numbers, then $x \geq 1$ or $y \geq 1$. \\
\setlength\parindent{24pt}The statement can be expressed as $p = $"if x + y $\geq$ 2" and $q = $"then $x \geq 1$ or $y \geq 1$"\\
\setlength\parindent{24pt} We first place the original $p \rightarrow q$ proposition in the form of $\neg q \rightarrow \neg p$.  \\
\setlength\parindent{24pt}$\neg q$ becomes "if x $<$ 1 and y $<$ 1" and $\neg p$ becomes "then x + y $<$ 2". \\
\setlength\parindent{24pt} If we add the inequalities of x $<$ 1 and y $<$ 1 together it becomes $x + y < 2$ which proves \\
the conditional statement $\neg q \rightarrow \neg p$. \\
\setlength\parindent{0pt}16) Prove that if $m$ and $n$ are integers and $mn$ is even, then $m$ is even or $n is even$. \\
\setlength\parindent{24pt}We will use a proof by contraposition.  For any integer k, the form 2k will yield an even number \\
and the form 2k+1 will yield an odd number.\\
\setlength\parindent{24pt} In standard form $p \rightarrow q$, p = "mn is even" and q = "m is even or n is even".\\
\setlength\parindent{24pt} We build the contrapositive of form $\neg q \rightarrow \neg p$.  $\neg q = $"m is odd AND n is odd." | $\neg p = $"mn is odd". \\ 
\setlength\parindent{24pt}We then set the equation $mn = (2k+1)(2j+1)$\\
\setlength\parindent{24pt}Multiply out the equation | $mn = 4kj + 2k + 2j + 1$.  Then factor out a 2 to yield:\\
\setlength\parindent{24pt}$mn = 2( 2kj + k + j ) + 1$ | Using the product of integers and sum of integers $( 2kj + k + j )$ we know \\
will be an integer. \\
\setlength\parindent{24pt} We now have the equation in the form $mn = 2(integer) + 1$ which is the aforementioned \\
for any integer k, 2k + 1 will be odd. \\
~\\
\setlength\parindent{0pt}17) Show that if $n$ is an integer and $n^3 + 5$ is odd, then $n$ is even using a proof by contraposition.\\
\setlength\parindent{24pt} For any integer k, 2k will yield an even number and 2k+1 will yield an odd number.\\
\setlength\parindent{24pt} We place the above statement is proper form of $\neg q \rightarrow \neg p$.  It reads ass $\neg q = $"If n is odd" \\
then $\neg p = $ then $n^3 + 5 $ is even.\\
\setlength\parindent{24pt} We then set $n = 2k + 1$.  $(2k + 1)(2k + 1)(2k + 1) + 5$\\
\setlength\parindent{24pt} Multiplied out it becomes $8k^3 + 12k^2 + 6k + 1 + 5$\\
\setlength\parindent{24pt} We then consolidate the 5 and 1 and factor a 2 out of the entire equation yielding:\\
\setlength\parindent{24pt} $2(4k^3 + 6k^2 + 3k + 3)$ | Using products of integers and sums of integers the entire \\
statement $(4k^3 + 6k^2 + 3k + 3)$ can be treated as an integer k. \\
\setlength\parindent{24pt} We are now in the form 2k or 2(integer).  Thus, when n is odd then $n^3 + 5$ is even.



\end{flushleft}
\end{document}