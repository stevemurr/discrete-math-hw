\documentclass{article}
\usepackage[margin=1in]{geometry}
\usepackage{graphicx}
\usepackage[fleqn]{amsmath}
\usepackage{color}
\usepackage{lipsum}
\begin{document}
\setcounter{totalnumber}{5}
   \begin{flushright}
      \Large\textbf{Steven Murr}\\
      \large\textit{HW 10.2} \\
      \large\textit{ Problems = \{ 1,2,3,4,7,8,12,20,24,25,35,40,42abc \} }
   \end{flushright}
\begin{flushleft}
\makeatletter% Set distance from top of page to first float
\setlength{\@fptop}{5pt}
\makeatother

\setlength\parindent{0pt}In exercises 1-3 find the number of vertices, the number of edges, and the degree of each vertex in the given undirected graph.  Identify all isolated and pendant vertices. \\
~\\
**See attached paper. \\
~\\
\setlength\parindent{0pt}4) Find the sum of the degrees of the vertices of each graph in Exercises 1-3 and verify that it equals twice the number of edges in the graph. \\
\setlength\parindent{24pt}Graph 1) $2+4+1+0+2+3 = 12$ and the $6 \cdot 2 = 12$ therefore the sum of the degrees has been \\verified. \\
\setlength\parindent{24pt}Graph 2) $6+6+6+5+3 = 26$ and $13 \cdot 2 = 26$ \\
\setlength\parindent{24pt}Graph 3) $3+2+4+0+6+0+4+2+3 = 24$ and $12 \cdot 2 = 24$ \\
~\\
\setlength\parindent{0pt} In Exercises 7-9 determine the number of vertices and edges and find the in-degree and out-degree of each vertex for the given directed multigraph. \\
~\\
**See attached paper. \\
~\\
\setlength\parindent{0pt}12) What does the degree of a vertex represent in the acquaintanceship graph, where vertices represent all the people in the world?  What does the neighborhood a vertex in this graph represent?  What do isolated and pendant vertices in this graph represent?  In one study it was estimated that the average degree of a vertex in this graph is 1000.  What does this mean in terms of the model? \\
\setlength\parindent{24pt} The degree of the vertex represents the number of acquaintances in relation to that specific vertex or node. \\
\setlength\parindent{24pt}The neighborhood of a vertex in this graph represents a subset of vertex's connected to any vertex.  (i.e. N(a) = { subset of other vertex's connected to a by an edge } \\
~\\
\setlength\parindent{24pt}Isolated vertices represent a person who has no acquaintances. \\
~\\
\setlength\parindent{24pt}A pendant vertex represents someone who has only one acquaintance.  \\
~\\
\setlength\parindent{24pt}If the average degree of a vertex is 1000 then that means, the average number of acquaintances in the graph is 1000 (i.e. there are on average 1000 edges connecting each vertex). \\
~\\
\setlength\parindent{0pt}20) Draw these graphs: \\
**See attached paper. \\
~\\
\setlength\parindent{24pt} In exercises 21 to 25 determine whether the graph is bipartite.  You may find it useful to apply Theorem 4. \\
**See attached paper. \\
~\\
\setlength\parindent{0pt}35) How many vertices and how many edges do these graphs have? \\
\setlength\parindent{24pt}a) $K_n$ \\
\setlength\parindent{24pt}Vertices: n, edges: n(n-1) / 2 \\
~\\
\setlength\parindent{24pt}b) $C_n$ \\
\setlength\parindent{24pt}Vertices: c, edges: c \\
~\\
\setlength\parindent{24pt}c) $W_n$ \\
\setlength\parindent{24pt}Vertices: n+1, edges: 2(n-1) \\
~\\
\setlength\parindent{24pt}d) $K_{m,n}$ \\
\setlength\parindent{24pt} Vertices: n+m, edges: mn \\
~\\
\setlength\parindent{24pt}e) $Q_n$ \\
\setlength\parindent{24pt}Vertices: $2^n$, edges: $2^{n-1}n$ \\
~\\
\setlength\parindent{0pt}40) How many edges does a graph have if its degree sequence is 4,3,3,2,2? Draw such a graph. \\
~\\
**See attached paper. \\


\end{flushleft}
\end{document}